\documentclass[12pt]{article}
\usepackage[utf8]{inputenc}
\usepackage[dvips]{graphicx}
\usepackage{epsfig}
\usepackage{fancybox}
\usepackage{verbatim}
\usepackage{array}
\usepackage{latexsym}
\usepackage{alltt}
\usepackage{amssymb}
\usepackage{amsmath}
\usepackage{hyperref}
\usepackage{listings}
\usepackage{color}
\usepackage{algorithm}
\usepackage{algpseudocode}
\usepackage[hmargin=3cm,vmargin=5.0cm]{geometry}
\usepackage{epstopdf}
\topmargin=-1.8cm
\addtolength{\textheight}{6.5cm}
\addtolength{\textwidth}{2.0cm}
\setlength{\oddsidemargin}{0.0cm}
\setlength{\evensidemargin}{0.0cm}
\newcommand{\HRule}{\rule{\linewidth}{1mm}}
\newcommand{\kutu}[2]{\framebox[#1mm]{\rule[-2mm]{0mm}{#2mm}}}
\newcommand{\gap}{ \\[1mm] }
\newcommand{\Q}{\raisebox{1.7pt}{$\scriptstyle\bigcirc$}}
\newcommand{\minus}{\scalebox{0.35}[1.0]{$-$}}

\usepackage{multicol}
\usepackage{multirow}


\lstset{
    %backgroundcolor=\color{lbcolor},
    tabsize=2,
    language=MATLAB,
    basicstyle=\footnotesize,
    numberstyle=\footnotesize,
    aboveskip={0.0\baselineskip},
    belowskip={0.0\baselineskip},
    columns=fixed,
    showstringspaces=false,
    breaklines=true,
    prebreak=\raisebox{0ex}[0ex][0ex]{\ensuremath{\hookleftarrow}},
    %frame=single,
    showtabs=false,
    showspaces=false,
    showstringspaces=false,
    identifierstyle=\ttfamily,
    keywordstyle=\color[rgb]{0,0,1},
    commentstyle=\color[rgb]{0.133,0.545,0.133},
    stringstyle=\color[rgb]{0.627,0.126,0.941},
}


\begin{document}

\noindent
\HRule %\\[3mm]
\small
\begin{center}
  \LARGE \textbf{CENG 483} \\[4mm]
  \Large Introduction to Computer Vision \\[4mm]
  \normalsize Fall 2023-2024 \\
  \Large Take Home Exam 1 \\
  \Large Instance Recognition with Color Histograms \\
    \Large Full Name: \textbf{TODO} \\
    \Large Student ID: \textbf{TODO} \\
\end{center}
\HRule

\begin{center}
\end{center}
\vspace{-10mm}
\noindent\\ \\ 
Please fill in the sections below only with the requested information. If you have additional things you want to mention, you can use the last section. For all of the configurations make sure that your quantization interval can divide 256 in order to obtain equal bins.

Remove all instructional text (such as this one) in your final submission.


\section{3D Color Histogram (RGB)}

In this section, don't divide the images into grids and don't convert them into HSV color space.
Your histogram must have at most 4096 bins. E.g. Assume that you choose 16 for quantization interval then you will have 16 bins for each channel and 4096 bins for your 3D color histogram.

\begin{table}[H]
  \centering
  \begin{tabular}{|l|l|l|l|}
    \hline
    \multirow{2}*{Q. Interval} & \multicolumn{3}{c|}{Query Set} \\
    \cline{2-4}
         & Query 1 & Query 2 & Query 3 \\
    \hline
    TODO & TODO    & TODO    & TODO    \\
    TODO & TODO    & TODO    & TODO    \\
    TODO & TODO    & TODO    & TODO    \\
    TODO & TODO    & TODO    & TODO    \\
    \hline
  \end{tabular}
  \caption{Top-1 accuracy results using 3D color histogram (RGB).}
\end{table}

\begin{itemize}
  \item Pick 4 different quantization intervals and give your top-1 accuracy results for each of them on every query dataset.
  \item Explain the differences in results and potential causes of them.
\end{itemize}

\section{3D Color Histogram (HSV)}

\begin{table}[H]
  \centering
  \begin{tabular}{|l|l|l|l|}
    \hline
    \multirow{2}*{Q. Interval} & \multicolumn{3}{c|}{Query Set} \\
    \cline{2-4}
         & Query 1 & Query 2 & Query 3 \\
    \hline
    TODO & TODO    & TODO    & TODO    \\
    TODO & TODO    & TODO    & TODO    \\
    TODO & TODO    & TODO    & TODO    \\
    TODO & TODO    & TODO    & TODO    \\
    \hline
  \end{tabular}
  \caption{Top-1 accuracy results using 3D color histogram (HSV).}
\end{table}

\begin{itemize}
  \item Repeat the previous experiment in the HSV color space.
  \item Compare the results with their RGB counterparts. Explain the differences in results and potential causes of them.
\end{itemize}

\section{Per-Channel Color Histogram (RGB)}

In this section, don't divide the images into grids and don't convert them into HSV color space.

\begin{table}[H]
  \centering
  \begin{tabular}{|l|l|l|l|}
    \hline
    \multirow{2}*{Q. Interval} & \multicolumn{3}{c|}{Query Set} \\
    \cline{2-4}
         & Query 1 & Query 2 & Query 3 \\
    \hline
    TODO & TODO    & TODO    & TODO    \\
    TODO & TODO    & TODO    & TODO    \\
    TODO & TODO    & TODO    & TODO    \\
    TODO & TODO    & TODO    & TODO    \\
    TODO & TODO    & TODO    & TODO    \\
    \hline
  \end{tabular}
  \caption{Top-1 accuracy results using per-channel color histogram (RGB).}
\end{table}

\begin{itemize}
  \item Pick 5 different quantization intervals and give your top-1 accuracy results for each of them on every query dataset.
  \item Explain the differences in results and potential causes of them.
  \item Compare the differences in results between 3D and per-channel color histograms. Comment on the potential causes.
\end{itemize}

\section{Per-Channel Color Histogram (HSV)}

\begin{table}[H]
  \centering
  \begin{tabular}{|l|l|l|l|}
    \hline
    \multirow{2}*{Q. Interval} & \multicolumn{3}{c|}{Query Set} \\
    \cline{2-4}
         & Query 1 & Query 2 & Query 3 \\
    \hline
    TODO & TODO    & TODO    & TODO    \\
    TODO & TODO    & TODO    & TODO    \\
    TODO & TODO    & TODO    & TODO    \\
    TODO & TODO    & TODO    & TODO    \\
    TODO & TODO    & TODO    & TODO    \\
    \hline
  \end{tabular}
  \caption{Top-1 accuracy results using per-channel color histogram (HSV).}
\end{table}

\begin{itemize}
  \item Repeat the previous experiment in the HSV color space.
  \item Compare the results with their RGB counterparts. Explain the differences in results and potential causes of them.
  \item You can also make 3D vs. per-channel comparison in this section instead of the RGB section.
\end{itemize}

\section*{Best Configuration}

Before starting the next section, please pick the best configuration according to the experiments above and continue with them. Fill the following fields accordingly:

\begin{itemize}
  \item Color space:
  \item Quantization interval for 3D color histogram:
  \item Quantization interval for per-channel color histogram:
\end{itemize}

\section{Grid Based Feature Extraction - Query set 1}
Give your top-1 accuracy for all of the configurations below. Note that $2 \times 2$ spatial grid means that the cell size is $48\times48$ pixels.

\begin{table}[H]
  \centering
  \begin{tabular}{|l|l|l|l|l|}
    \hline
    \multirow{2}*{Histogram Type} & \multicolumn{4}{c|}{Spatial Grid} \\
    \cline{2-5}
                & $2 \times 2$ & $4 \times 4$ & $6 \times 6$ & $8 \times 8$ \\
    \hline
    3D          & TODO         & TODO         & TODO         & TODO         \\
    Per-Channel & TODO         & TODO         & TODO         & TODO         \\
    \hline
  \end{tabular}
  \caption{Top-1 accuracy results on query set 1.}
\end{table}

\subsection{Questions}
\begin{itemize}
\item What do you think about the cause of the difference between the results?
\item Explain the advantages/disadvantages of using grids in both types of histograms if there are any.
\end{itemize}

\section{Grid Based Feature Extraction - Query set 2}
Give your top-1 accuracy for all of the configurations below. Note that $2 \times 2$ spatial grid means that the cell size is $48\times48$ pixels.

\begin{table}[H]
  \centering
  \begin{tabular}{|l|l|l|l|l|}
    \hline
    \multirow{2}*{Histogram Type} & \multicolumn{4}{c|}{Spatial Grid} \\
    \cline{2-5}
                & $2 \times 2$ & $4 \times 4$ & $6 \times 6$ & $8 \times 8$ \\
    \hline
    3D          & TODO         & TODO         & TODO         & TODO         \\
    Per-Channel & TODO         & TODO         & TODO         & TODO         \\
    \hline
  \end{tabular}
  \caption{Top-1 accuracy results on query set 2.}
\end{table}

\subsection{Questions}
\begin{itemize}
\item What do you think about the cause of the difference between the results?
\item Explain the advantages/disadvantages of using grids in both types of histograms if there are any.
\end{itemize}


\section{Grid Based Feature Extraction - Query set 3}
Give your top-1 accuracy for all of the configurations below. Note that $2 \times 2$ spatial grid means that the cell size is $48\times48$ pixels.

\begin{table}[H]
  \centering
  \begin{tabular}{|l|l|l|l|l|}
    \hline
    \multirow{2}*{Histogram Type} & \multicolumn{4}{c|}{Spatial Grid} \\
    \cline{2-5}
                & $2 \times 2$ & $4 \times 4$ & $6 \times 6$ & $8 \times 8$ \\
    \hline
    3D          & TODO         & TODO         & TODO         & TODO         \\
    Per-Channel & TODO         & TODO         & TODO         & TODO         \\
    \hline
  \end{tabular}
  \caption{Top-1 accuracy results on query set 3.}
\end{table}

\subsection{Questions}
\begin{itemize}
\item What do you think about the cause of the difference between the results?
\item Explain the advantages/disadvantages of using grids in both types of histograms if there are any.
\end{itemize}


\section{Additional Comments and References}

(if there any)





\end{document}
